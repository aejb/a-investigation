% MathMode: plain, MathRender: svg, MathDpi: 300, MathEmbedLimit: 524288, MathScale: 105, MathBaseline: 0, MathDocClass: [10pt]book, MathImgDir: math, MathLatex: latex, MathSvgFontFormat: "", MathSvgSharePaths: True, MathSvgPrecision: 3, Dvisvg: dvisvgm
\documentclass[10pt]{book}
% generated by Madoko, version 1.1.6
%mdk-data-line={1}
\newcommand\mdmathmode{plain}
\newcommand\mdmathrender{svg}
\usepackage[heading-base={2},section-num={false},bib-label={true},fontspec={true}]{madoko2}


\begin{document}


\begin{mdSnippets}
%mdk-data-line={6}
%mdk-data-line={70}
\begin{mdInlineSnippet}[54c828d34a9ab5de03d1bd36d06f8bfe]%mdk
$I\propto\cos^2\theta$\end{mdInlineSnippet}%mdk
%mdk-data-line={72}
\begin{mdInlineSnippet}[3988589845bbb41b96d667f353b3d1bb]%mdk
$I=I_0\cos^2\theta$\end{mdInlineSnippet}%mdk
%mdk-data-line={74}
\begin{mdInlineSnippet}[053a5864c12b37b85c25fc88ddcedfa4]%mdk
$\Rightarrow y=mx+c$\end{mdInlineSnippet}%mdk
%mdk-data-line={76}
\begin{mdInlineSnippet}[65261bf0f42cf0f16dcf6f6742361736]%mdk
$\therefore y=I, m=I_0, x=\cos^2\theta, c=0$\end{mdInlineSnippet}%mdk
%mdk-data-line={80}

\end{mdSnippets}

\end{document}
